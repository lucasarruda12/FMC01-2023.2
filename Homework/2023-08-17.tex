\documentclass[a4paper, 12pt]{article}

% PACOTES
\usepackage[utf8]{inputenc} % Caracteres em geral
\usepackage[brazilian]{babel} % Idioma da página
\usepackage[top=2cm, bottom=2cm, left=2cm, right=2cm]{geometry} % margens
\usepackage{fancyhdr} % Cabeçalho
\usepackage{ulem} % Destacar, sublinhar, riscar e etc...
\usepackage{graphicx} % Figuras
\usepackage{booktabs} % Tabelas
\usepackage{multicol} % Multicolunas para tabelas
\usepackage{multirow} % Multilinhas para tabelas
\usepackage{colortbl} % Colorir tabelas
\usepackage[table]{xcolor} % Para usar !25 !50 !75 nas cores
\usepackage{float} % Para manter as tabelas no canto certo
\usepackage{amsmath} % Matrizes
\usepackage{hyperref} % Hyperlinks
\usepackage{tikz} % TikZ - Desenhos e Gráficos
\usepackage{amsmath,amssymb} % Mais símbolos matemáticos
\usepackage[shortlabels]{enumitem} % Usar letras no enumerate

% PREÂMBULOS
\pagenumbering{Roman}
\tikzstyle{every picture}+=[remember picture,inner xsep=0,inner ysep=0.25ex]

% INÍCIO DO DOCUMENTO
\begin{document}

\section{HOMEWORK}
\subsection{2023-08-16}
\textbf{EXERCÍCIO x1.1 \\
Já que ‘ $\iff$ ’ corresponde à frase $\ll$se e somente se$\gg$, faz sentido pensar que uma das setinhas envolvidas (‘ $\implies$ ’ e ‘ $\impliedby$ ’) corresponde na frase $\ll$se$\gg$ e a outra na $\ll$somente se$\gg$. Qual é qual?} \\ \\
Na minha concepção, $\impliedby$ é equivalente a "se", e $\implies$ é equivalente a "somente se". Isso porque: \\ \\
Se supormos duas proposições, a e b. \\ \\
Ao dizermos: "a, se b", estamos afirmando que a partir da proposição b, nós podemos determinar a veracidade da proposição a, ou seja, b $\implies$ a (ou a $\impliedby$ b). \\ \\
Da mesma forma, ao dizermos: "a, somente se b", estamos afirmando que a partir da proposição a, nós podemos determinar a veracidade da proposição b, ou seja, a $\implies$ b. \\ \\
Um exemplo: \\ \\
Sejam as proposições a e b, respectivamente, "é brasileiro" e "gosta de futebol" \\ \\
Se concordarmos que alguém \textcolor{red}{é brasileiro se gosta de futebol}, gostar de futebol passa a ser suficiente para determinar se alguém é ou não brasileiro. E, portando, gostar ou não de futebol implica ($\implies$) ser ou não brasileiro. \\ \\
Da mesma forma, ao concordarmos que alguém \textcolor{red}{é brasileiro somente se gosta de futebol}, ser brasileiro torna-se suficiente para determinar a opinião de alguém sobre futebol. Ou seja, ser ou não brasileiro implica ($\implies$) gostar ou não de futebol. \\ \\

\noindent \textbf{EXERCÍCIO x1.2. \\
Mesma pergunta, agora lendo o ‘ $\iff$ ’ como $\ll$é suficiente e necessário para$\gg$. Uma direção corresponde ao  $\ll$suficiente$\gg$ outra ao  $\ll$necessário$\gg$. Qual é qual?} \\ \\
Na minha concepção, $\implies$ é equivalente a "é suficiente para", e $\implies$ é equivalente a "é necessário para". Isso porque: \\ \\
Se supormos duas proposições, a e b. \\ \\
Ao dizermos: "a é necessário para b", nós podemos assumir que a segunda é uma condição da primeira. Logo, se b é verdadeiro, todas as suas condições, incluindo a, são verdadeiras, portando, b $\implies$ a (ou a $\impliedby$ b). \\ \\
Da mesma forma, ao dizermos: "a é suficiente b", nós podemos assumir que a primeira é a única condição de existência da segunda e, portanto, a $\implies$ b. \\ \\
Um exemplo: \\ \\
Sejam as proposições a e b, respectivamente, "ser alto" e "ser bonito". \\ \\
Se concordamos que \textcolor{red}{ser alto é necessário para ser bonito}, podemos assumir todo homem bonito é alto, pois ser alto é uma condição para ser bonito. E, portanto, ser bonito implica ($\implies$) ser alto. \\ \\
E, se concordamos que \textcolor{red}{ser alto é suficiente para ser bonito}, podemos assumir que todo homem alto é bonito. E, portando, ser alto implica ($\implies$) ser bonito. \\ \\

\noindent \textbf{EXERCÍCIO x1.3. \\
Mude cada uma das frases do Exemplo 1.9 para resolver o type error. Não se preocupe
com a veracidade.} \\ \\
(1) $(x + y)^{2} \textcolor{red}{~=~} x^{2} + 2xy + y^{2}$. \\
(2) $x(a - (b + c)) = xa - x(b + c) \textcolor{red}{~=~} xa - xb - xc$. \\
(3) Concluimos que $(A \subseteq B) \textcolor{red}{~\implies~} (B \subseteq A)$. \\
(4) Suponha n. Vamos demonstrar que n + 1 \textcolor{red}{= r}. \\ \\


\noindent \textbf{EXERCÍCIO x1.4. \\ 
Considere a definição seguinte: \\ \\
Definição. Seja n um inteiro. Chamamos o n de par se e somente se n = 2k para qualquer inteiro k. \\ \\
Essa definição “compila”. Mas o conceito que foi definido não é o que o seu escritor tinha
no coração dele. \\ \\
Por quê? Qual o problema com a definição do Exemplo 1.25? Como podemos consertar?} \\ \\
A falta de pontuação na definição cria um sentido diferente do que o autor provavelmente intencionava. \\ \\
Pontuando corretamente esse texto, de forma a torná-lo menos ambíguo, teríamos: "Seja n um inteiro. Chamamos o n de par se, e somente se, n = 2k, para qualquer inteiro k." \\ \\

\noindent \textbf{EXERCÍCIO x1.5. \\
Verdade ou falso?: \\ \\
(i) se $A \equiv B$ então $A = B$; \\
(ii) se $A \Lleftarrow\Rrightarrow B$ então $A \iff B$. \\ \\
Observe que para cada uma dessas afirmações fazer sentido os A, B denotam objetos na primeira, mas proposições na segunda. } \\ \\
Ambas as proposições são verdadeiras. \\ \\

\noindent \textbf{EXERCÍCIO x1.6. \\
Para cada um par de expressões escolha a melhor opção dos:
$\Lleftarrow\Rrightarrow$, $\iff$, $\equiv$, =. \\ \\
Observe que as versões intensionais são mais fortes que as extensionais, então quando aplica a versão intensional, precisa escolhé-la.} \\
\begin{enumerate}
    \item $\left\{
    \begin{aligned}
    &2\times3 \\
    &6
    \end{aligned}
    \right.$ \\ \\
    Nesse caso, eu usaria o símbolo de =, uma vez que os dois membros são números, são iguais, mas não necessariamente são intencionalmente iguais. \\

    \item $\left\{
    \begin{aligned}
    &2\times3 \\
    &3\times2
    \end{aligned}
    \right.$ \\ \\
    Novamente, eu usaria o símbolo de =. Ambos os membros são números mas intencionam coisas diferentes. \\

    \item $\left\{
    \begin{aligned}
    &\textrm{x ama y}\\
    &\textrm{y é amado por x}
    \end{aligned}
    \right.$ \\ \\
    Seguimos aqui a mesma lógica da alternativa anterior, porém, agora temos proposições. Sendo assim, eu usaria $\iff$. \\

    \item $\left\{
    \begin{aligned}
    &\textrm{n é par}\\
    &\textrm{existe } k \in \mathbb{Z} \textrm{ tal que } n = 2k
    \end{aligned}
    \right.$ \\ \\
    Eu usaria $\Lleftarrow\Rrightarrow$. Veja que são duas proposções (tem verbo) e uma é a definição da outra. \\

    \item $\left\{
    \begin{aligned}
    &\textrm{Matheus mora na capital do RN}\\
    &\textrm{Matheus mora na maior cidade do RN}
    \end{aligned}
    \right.$ \\ \\
    Eu usaria $\iff$. Veja que são duas proposções (tem verbo), mas ser a maior cidade não é sinônimo de ser capital. \\

    \item $\left\{
    \begin{aligned}
    &\textrm{a capital da Grécia}\\
    &\textrm{Aténas}
    \end{aligned}
    \right.$ \\ \\
    Eu usaria =, pois são dois objetos e, apesar de Aténas ser a capital da Grécia, ser capital da Grécia não é sinônimo de ser Aténas. \\

    \item $\left\{
    \begin{aligned}
    &\textrm{o vocalista da banda Sarcófago é professor da UFMG}\\
    &\textrm{Wagner Moura é professor da maior universidade de MG}
    \end{aligned}
    \right.$ \\ \\
    $\iff$. São proposições. Não tem a mesma intensionalidade. \\

    \item $\left\{
    \begin{aligned}
    &\textrm{Aristoteles foi professor de Alexandre o Grande
    }\\
    &\textrm{Aristoteles ensinou Alexandre o Grande}
    \end{aligned}
    \right.$ \\ \\
    $\Lleftarrow\Rrightarrow$. Ambas são proposições. Ser professor tem exatamente o mesmo sentido de ensinar. \\

    \item $\left\{
    \begin{aligned}
    &\textrm{A terra é plana}\\
    &\textrm{A lua é feita de queijo}
    \end{aligned}
    \right.$ \\ \\
    $\iff$. Ambas são proposições. Não tem a mesma intensionalidade \\
    
    \item $\left\{
    \begin{aligned}
    &x^{2} + y^{2} \leq 0\\
    &x=y=0
    \end{aligned}
    \right.$ \\ \\
    $\iff$. Ambas são proposições. Não tem a mesma intensionalidade. \\

    \item $\left\{
    \begin{aligned}
    &x^{2} + y^{2} \leq 0\\
    &0 \geq x \times x + y \times y
    \end{aligned}
    \right.$ \\ \\
    $\iff$. Ambas são proposições. Apesar de $x \times x$ ser a definição de $x^{2}$, eu acredito que cai no mesmo caso que a segunda alternativa, por causa da mudança no operador (do $\leq$ para $\geq$) \\

    \item $\left\{
    \begin{aligned}
    &(x^{2} + y^{2})^{2} \\
    &(x \times x + y \times y) \times (x \times x + y \times y)
    \end{aligned}
    \right.$ \\ \\
    $\equiv$. São objetos(números). Veja que $x \times x$ é simplesmente a definição de $x^{2}$, e ambos intencionam a mesma coisa. \\

\end{enumerate}

\noindent \textbf{EXERCÍCIO x1.7 (typecheck warmup). \\
Considere as expressões abaixo. Para cada uma decida se ela denota objeto ou proposição.}

\begin{enumerate}[(a)]
    \item $(x + y)^{2} = x^{2} + 2xy$ \textcolor{red}{$\rightarrow$ é uma proposição, pois tem "verbo".}
    \item a mãe de $p$ \textcolor{red}{$\rightarrow$ É um objeto, pois denota um indivíduo e não tem "verbo".}
    \item $2^{n} + 1$ \textcolor{red}{$\rightarrow$ É um objeto, pois denota um número e não tem "verbo".}
    \item $p$ é irmão de $q$ \textcolor{red}{$\rightarrow$ Proposição.}
    \item a capital do país $p$ \textcolor{red}{$\rightarrow$ Objeto.}
    \item $a$ mora em Atenas \textcolor{red}{$\rightarrow$ Proposição.} \\
\end{enumerate}

\noindent \textbf{EXERCÍCIO x1.8. \\
Para cada uma das cinco expressões do 1.35: objeto ou proposição?}
\begin{enumerate}[(a)]
    \item existe $k \in \mathbb{Z}$ tal que $13 = 2k + 1$ \textcolor{red}{$\rightarrow$ Proposição, por conta do verbo "existe".}
    \item existem números $x,y$ tais que $(x+y)^{2} = x^{2}+2xy$ \textcolor{red}{$\rightarrow$ Proposição.}
    \item aquela função que dada um número $x$ retorna o $x+1$ \textcolor{red}{$\rightarrow$ A sentença define uma função. Resta saber se uma função é um objeto ou uma proposição. Eu imagino que seja um objeto.}
    \item o conjunto de todos livros $b$ tais que existe palavra $w$ no $b$ com tantas letras quantas as letras do título de $b$ \textcolor{red}{$\rightarrow$ Objeto.}
    \item para toda pessoa $p$, a pessoa $q$ não gosta de $p$. \textcolor{red}{$\rightarrow$ Proposição.} \\
\end{enumerate}

\noindent \textbf{EXERCÍCIO x1.9.
Enuncie cada uma das $(1)-(4)$ do 1.35 sem pronunciar nenhuma das variáveis ligadas
que aparecem. }
\begin{enumerate}[(a)]
    \item existe $k \in \mathbb{Z}$ tal que $13 = 2k + 1$ \textcolor{red}{$\rightarrow$ 13 é igual a duas vezes algum número inteiro mais um.}
    \item existem números $x,y$ tais que $(x+y)^{2} = x^{2}+2xy$ \textcolor{red}{$\rightarrow$ O quadrado da soma de algum par de números é igual a soma do quadrado de um com duas vezes o produto dos dois.}
    \item aquela função que dada um número $x$ retorna o $x+1$ \textcolor{red}{$\rightarrow$ Aquela função que retorna um número mais um.}
    \item o conjunto de todos livros $b$ tais que existe palavra $w$ no $b$ com tantas letras quantas as letras do título de $b$ \textcolor{red}{$\rightarrow$ O conjunto dos livros que tem uma palavra com o mesmo número de letras que o seu título.}
    \item para toda pessoa $p$, a pessoa $q$ não gosta de $p$. \textcolor{red}{$\rightarrow$ $q$ não gosta de ninguém.} \\
\end{enumerate}

\noindent \textbf{EXERCÍCIO x1.10. \\
Mesma coisa sobre as frases seguintes:}

\begin{enumerate}[(a)]
    \item existem pessoas $p,q$ tais que $p$ ama $q$ e $q$ ama $p$ \textcolor{red}{$\rightarrow$ Existe um par de pessoas que se amam.}
    \item existe pessoa $p$ tal que $p$ ama $q$ e $q$ ama $p$ \textcolor{red}{$\rightarrow$ $q$ ama e é amado por alguém.}
    \item $x+y=z$ \textcolor{red}{$\rightarrow$ $x+y=z$, todas as variáveis são livres}
    \item existe um número $x$ tal que $x + y = z$ \textcolor{red}{$\rightarrow$ $y$ somado a algum número é igual a $z$.}
    \item existem números $x,z$ tais que $x + y = z$ \textcolor{red}{$\rightarrow$ $y$ somado a algum número é igual a algum outro número.}
    \item para qualquer número $y$, existe um número $x$ tal que $x + y = z$ \textcolor{red}{$\rightarrow$ Para todo número, existe outro tal que a soma dos dois é igual a $z$}
    \item para quaisquer números $y,z$, existe número $x$ tal que $x + y = z$ \textcolor{red}{$\rightarrow$ Para qualquer par de números, existe um valor tal que o primeiro número do par somado a ele é igual ao segundo.}
\end{enumerate}

\noindent \textbf{EXERCÍCIO x1.11. \\
Desenha as ligações da $(3)$ e $(5)$ do Exemplo 1.47.} \\ \\
%
$(3)$ existe \tikz[baseline=(node1.base)]\node (node1)  {$\bullet$};  tal que \tikz[baseline=(node2.base)]\node (node2) {$\bullet$}; $-~d$ e \tikz[baseline=(node3.base)]\node (node3) {$\bullet$}; $+~d$ são primos. 
%
\begin{tikzpicture}[overlay]
    \draw[-latex] (node1.south) to[bend right] (node2.south);
    \draw[-latex] (node1.south) to[bend right] (node3.south);
\end{tikzpicture} \\ \\
%
Existe \tikz[baseline=(node1.base)]\node (node1)  {$\bullet$}; tal que para todo \tikz[baseline=(node3.base)]\node (node3)  {\textcolor{red}{$\bullet$}}; , se \tikz[baseline=(node4.base)]\node (node4)  {\textcolor{red}{$\bullet$}}; $\geq$ \tikz[baseline=(node2.base)]\node (node2)  {$\bullet$};, então existe \tikz[baseline=(node7.base)]\node (node7)  {\textcolor{blue}{$\bullet$}}; tal que \tikz[baseline=(node5.base)]\node (node5)  {\textcolor{red}{$\bullet$}}; - \tikz[baseline=(node8.base)]\node (node8)  {\textcolor{blue}{$\bullet$}}; e \tikz[baseline=(node6.base)]\node (node6)  {\textcolor{red}{$\bullet$}}; + \tikz[baseline=(node9.base)]\node (node9)  {\textcolor{blue}{$\bullet$}}; são primos 
%
\begin{tikzpicture}[overlay]
    \draw[-latex] (node1.south) to[bend right] (node2.south);
    \draw[-latex][red] (node3.south) to[bend right] (node4.south);
    \draw[-latex][red] (node3.south) to[bend right] (node5.south);
    \draw[-latex][red] (node3.south) to[bend right] (node6.south);
    \draw[-latex][blue] (node7.south) to[bend right] (node8.south);
    \draw[-latex][blue] (node7.south) to[bend right] (node9.south);
\end{tikzpicture}
\vspace{2cm}

\noindent \textbf{EXERCÍCIO x1.12 \\
Já encontramos dois ligadores $\ll$existe \_\_ tal que . . .$\gg$ e  $\ll$para todo \_\_, . . .  $\gg$. Dê mais
exemplos de ligadores que tu conhece: de matemática, de programação, de vida. . . } \\ \\
Eu penso em: Para todo\_\_, todos os\_\_, dado um\_\_ e seja\_\_. \\

\noindent \textbf{EXERCÍCIO x1.13. \\
Na $(3)$ do Exemplo 1.47 podemos renomiar a variável. . . :}
\begin{enumerate}[(i)]
    \item $'n'$ por $'m'$? \textcolor{red}{$\rightarrow$ Sim! $n$ é uma variável ligada.}
    \item $'n'$ por $'d'$? \textcolor{red}{$\rightarrow$ Não. O texto ficaria ambíguo.}
    \item $'d'$ por $'x'$? \textcolor{red}{$\rightarrow$ Não. $d$ é uma variável livre.}
\end{enumerate}

\noindent \textbf{Na (5) do Exemplo 1.47 podemos renomiar a variável. . . :} \\ \\
Poderíamos renomear todas as variáveis, pois são ligadas. Mas veja, se já temos uma variável chamada $N$, não faz sentido tentar dar o mesmo nome de $N$ a outra coisa. \\

\noindent \textbf{EXERCÍCIO x1.14 \\
Quais são esses $x \in P$ tais que sister($x$) não é determinado? Cuidado: tem mais que umacategoria de $x$'s “problemáticos”!} \\ \\
Bom, podemos supor uma pessoa (um person) que não tem irmã. Nesse caso, se tentássemos aplicar essa pessoa na função f, não teríamos o que retornar e sister seria indeterminado. \\ \\
Ou uma pessoa que tem mais de uma irmã. Nesse outro caso teríamos que retornar mais de um objeto para uma única entrada da função. \\ \\
Ou, ainda, uma pessoa que tenha uma irmã que não é da classe person, novamente não teríamos o que retornar. \\ \\
A função só seria valída se pudéssemos assumir que todas as pessoas tem uma única irmã. \\ \\

\noindent \textbf{EXERCÍCIO x1.15. \\
Defina o que significa ser parallelas} \\ \\
Retas são paralelas quando não dividem nenhum ponto entre si.\\ \\
Segmentos de retas são paralelos quando as retas das quais com as quais cada um divide pelo menos dois pontos são paralelas. \\ \\

\noindent \textbf{EXERCÍCIO x1.16. \\
Sejam $a, b, c$ números naturais. Usando $=$ para igualdade semântica e $\equiv$ para igualdade
sintáctica, decida para cada uma das afirmações seguintes se é verdadeira ou falsa:}

\begin{enumerate}[($i$)]
    \item $a + b + c \equiv a + (b + c)$ \textcolor{red}{$\rightarrow$ falso.}
    \item $a + b + c \equiv (a + b) + c$ \textcolor{red}{$\rightarrow$ verdadeiro.}
    \item $a + b + c = a + (b + c)$ \textcolor{red}{$\rightarrow$ verdadeiro.}
    \item $a + b + c = (a + b) + c$ \textcolor{red}{$\rightarrow$ verdadeiro.}
    \item $2 \cdot 0 + 3 = 0 + 3$ \textcolor{red}{$\rightarrow$ verdadeiro.}
    \item $2 \cdot 0 + 3 \equiv 0 + 3$ \textcolor{red}{$\rightarrow$ falso.}
    \item $(2 \cdot 0) + 3 \equiv 1 + 1 + 1$ \textcolor{red}{$\rightarrow$ falso.}
    \item $2 \cdot 0 + 3 \equiv 1 + 1 + 1$ \textcolor{red}{$\rightarrow$ falso.}
    \item $2 \cdot 0 + 3 \equiv 2 \cdot (0 + 3)$ \textcolor{red}{$\rightarrow$ falso.}
    \item $2 \cdot 0 + 3 = 2 \cdot (0 + 3)$ \textcolor{red}{$\rightarrow$ falso.}
    \item $2 \cdot 0 + 3 \equiv (2 \cdot 0) + 3$ \textcolor{red}{$\rightarrow$ verdadeiro.}
    \item $1 + 2 \equiv 2 + 1$ \textcolor{red}{$\rightarrow$ falso.}
\end{enumerate}
\vspace{1cm}

\noindent \textbf{EXERCÍCIO x1.17. \\
Verdade ou falso?: $A \equiv B \implies A = B$
} \\ \\
Verdade. \\ \\

\noindent \textbf{EXERCÍCIO x1.18. \\
Tente gerar a expressão
$(1 + 5) \cdot 2$
usando a Gramática 4.75.} \\ \\
\begin{align*}
    \langle ArEx \rangle &\stackrel{(0)}{\rightsquigarrow} \langle OpEx \rangle \\
    &\stackrel{(2)}{\rightsquigarrow} 
    ( \langle ArEx \rangle
    \langle BinOp \rangle
    \langle ArEx \rangle ) \\
    &\stackrel{(0)}{\rightsquigarrow} 
   ( \langle OpEx \rangle
    \langle BinOp \rangle
    \langle Num \rangle
    ) \\
    &\stackrel{(2)}{\rightsquigarrow} 
    ( ( \langle ArEx \rangle
    \langle BinOp \rangle
    \langle ArEx \rangle )
    \langle BinOp \rangle
    \langle Num \rangle
    ) \\
    &\stackrel{(1)}{\rightsquigarrow} 
    ( ( \langle Num \rangle
    \langle BinOp \rangle
    \langle Num \rangle )
    \langle BinOp \rangle
    \langle Num \rangle
    ) \\
    &\stackrel{(3)}{\rightsquigarrow} 
    ( ( \langle Num \rangle
    +
    \langle Num \rangle )
    \cdot
    \langle Num \rangle
    ) \\
    &\stackrel{(1)}{\rightsquigarrow} 
    ( (1+5)\cdot 2) \equiv (1+5)\cdot 2
\end{align*}

\noindent \textbf{EXERCÍCIO x1.19. \\
Mostre como um while loop pode ser implementado como açúcar sintáctico numa linguagem que tem for loops mas não while loops, e vice versa.} \\ \\
\begin{enumerate}
    \item De for para while:

    \begin{verbatim}
    for (;;){
        if (condicao){
            break;
        }

        // CÓDIGO
    }
    \end{verbatim}
    \textcolor{red}{A ideia principal é montar o um for loop que roda indefinidamente, até que o ciclo seja quebrado quando a condição for cumprida.}

    \item De while para for:
    
    \begin{verbatim}
    int contador;

    while(1){
        if (contador satisfaz condição){
            break;
        }

        (o que acontece com o contador a cada iteração)

        \\ CÓDIGO
    }
    \end{verbatim}

    \textcolor{red}{De novo a ideia é criar um loop infinito que quebra somente quando a condição é satisfeita. Mas, dessa vez, com uma variável chamada contador que é incrementada ou decrementada (ou sofre qualquer outra alteração, inclusive nenhuma alteração) a cada iteração do loop.}
        
\end{enumerate}
\vspace{1cm}
\noindent \textbf{EXERCÍCIO x1.20
Mostre como um for loop no estilo da linguagem C pode ser implementado sem usar
nenhum dos loops disponíveis em C (for, while, do-while).}
Eu faria algo assim:
\begin{verbatim}
int for_loop(int:contador, prop condicao, cmd operacao){
    \\ CÓDIGO

    if(condicao){
        return 0;
    }

    operacao;

    for_loop(contador, condicao, operacao);
}
\end{verbatim}
\textcolor{red}{Usar recursão foi a única forma que eu encontrei de rodar essas linhas de código em loop (já que foi proíbido usar goto :/). O único problema que eu encontrei para fazer esse código rodar em C exatamente igual a um for loop é uma forma de passar uma proposição (como contador $\leq$ 0) como argumento da função.}


\end{document}

% ANOTAÇÕES

% \\ ou \newline -> quebra de linha
% \vspace{1cm} -> espaçamento vertical
% \hspace{1cm} -> espaçamento horizontal
% ~ -> equivalente a um espaço
% caracteres especiais -> \# \$ \& \_ \{ \} \textbackslash \textasciitilde
% LaTeX table generator