\documentclass[a4paper, 12pt]{article}

% PACOTES
\usepackage[utf8]{inputenc} % Caracteres em geral
\usepackage[brazilian]{babel} % Idioma da página
\usepackage[top=2cm, bottom=2cm, left=2cm, right=2cm]{geometry} % margens
\usepackage{fancyhdr} % Cabeçalho
\usepackage{ulem} % Destacar, sublinhar, riscar e etc...
\usepackage{graphicx} % Figuras
\usepackage{booktabs} % Tabelas
\usepackage{multicol} % Multicolunas para tabelas
\usepackage{multirow} % Multilinhas para tabelas
\usepackage{colortbl} % Colorir tabelas
\usepackage[table]{xcolor} % Para usar !25 !50 !75 nas cores
\usepackage{float} % Para manter as tabelas no canto certo
\usepackage{amsmath} % Matrizes
\usepackage{hyperref} % Hyperlinks
\usepackage{tikz} % TikZ - Desenhos e Gráficos
\usepackage{amsmath,amssymb} % Mais símbolos matemáticos
\usepackage[shortlabels]{enumitem} % Usar letras no enumerate

% PREÂMBULOS
\pagenumbering{Roman}
\tikzstyle{every picture}+=[remember picture,inner xsep=0,inner ysep=0.25ex]

% INÍCIO DO DOCUMENTO
\begin{document}

\section{HOMEWORK}
\subsection{2023-08-27}

\textbf{1. Infira a simetria e a transitividade da igualdade a partir das reflexividade e substituição.}
\begin{enumerate}[a.]
    \item Simetria
    \begin{table}[h!]
        \rowcolors{2}{gray!25}{white}
        \centering
        \begin{tabular}{|p{4cm} | p{4cm} | p{4cm} |}
        \rowcolor{gray!50}
            Demonstração & Dados & Alvo \\
            && $a = b \implies b = a$  \\
            Suponha $a = b$ & $a = b$ & $b = a$\\
            Substitua $a$ por $b$ no alvo usando $a = b$ & $a=b$&$b=b$\\
            Substitua $a$ por $b$ em $a = b$ usando $a=b$ & $a=b$, $b=b$&$b=b$\\
            Imediato &&\\
            \hline
            \end{tabular}
    \end{table}

    \item Transitividade
    \begin{table}[h!]
        \rowcolors{2}{gray!25}{white}
        \centering
        \begin{tabular}{|p{4cm} | p{4cm} | p{4cm} |}
        \rowcolor{gray!50}
            Demonstração & Dados & Alvo \\
            && $(a = b) \& (b = c) \implies (a = c)$  \\
            Suponha $(a = b) \& (b =c)$ & $(a = b) \& (b =c)$ & $a=c$\\
            ext-L de $(a = b) \& (b =c)$ & $(a = b) \& (b =c)$, $a=b$ & $a = c$\\
            ext-R de $(a = b) \& (b =c)$ & $(a = b) \& (b =c)$, $a=b$, $b=c$ & $a=c$\\
            Substitua $b$ por $c$ em $a=b$ & $(a = b) \& (b =c)$, $a=b$, $b=c$, $a=c$ & $a=c$\\
            Imediato & & \\
            \hline
            \end{tabular}
    \end{table}
\end{enumerate}

\end{document}

% ANOTAÇÕES

% \\ ou \newline -> quebra de linha
% \vspace{1cm} -> espaçamento vertical
% \hspace{1cm} -> espaçamento horizontal
% ~ -> equivalente a um espaço
% caracteres especiais -> \# \$ \& \_ \{ \} \textbackslash \textasciitilde
% LaTeX table generator