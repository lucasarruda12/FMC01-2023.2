\documentclass[a4paper, 12pt]{article}

% PACOTES
\usepackage[utf8]{inputenc} % Caracteres em geral
\usepackage[brazilian]{babel} % Idioma da página
\usepackage[top=2cm, bottom=2cm, left=2cm, right=2cm]{geometry} % margens
\usepackage{fancyhdr} % Cabeçalho
\usepackage{ulem} % Destacar, sublinhar, riscar e etc...
\usepackage{graphicx} % Figuras
\usepackage{booktabs} % Tabelas
\usepackage{multicol} % Multicolunas para tabelas
\usepackage{multirow} % Multilinhas para tabelas
\usepackage{colortbl} % Colorir tabelas
\usepackage[table]{xcolor} % Para usar !25 !50 !75 nas cores
\usepackage{float} % Para manter as tabelas no canto certo
\usepackage{amsmath} % Matrizes
\usepackage{hyperref} % Hyperlinks
\usepackage{tikz} % TikZ - Desenhos e Gráficos
\usepackage{amsmath,amssymb} % Mais símbolos matemáticos
\usepackage[shortlabels]{enumitem} % Usar letras no enumerate

% PREÂMBULOS
\newcounter{linhadem}
\setcounter{linhadem}{0}

\newenvironment{dem}
{
    \rowcolors{2}{gray!25}{white}
    \centering
    \begin{tabular}{p{0.3cm} |p{8cm} |}
    \rowcolor{gray!50}
     & Demonstração\\
}
{
  \end{tabular}
}

\newenvironment{alvodados}
{
    \centering
    \begin{tabular}{p{2.9cm} |p{2.9cm}}
    Dados & Alvo\\
    \hline
}
{
  \end{tabular}
  \setcounter{linhadem}{0}
}

\newcommand{\linha}[1]{%
  \stepcounter{linhadem}
  \thelinhadem & #1 \\
}

\newcommand{\cmnt}[2]{%
  #1 & #2 \\
}

\pagenumbering{Roman}
\tikzstyle{every picture}+=[remember picture,inner xsep=0,inner ysep=0.25ex]

% INÍCIO DO DOCUMENTO
\begin{document}

\section{2023-09-04}

\subsection*{Capítulo 3: até $\theta$3.13}
%
\subsubsection*{EXERCÍCIO x3.1.\\
Para cada axioma que parece favorecendo o lado direito, sua versão esquerda é um
teorema: o 0 é uma (+)-identidade-L; o 1 é uma (·)-identidade-L; para todo a o -a é um
(+)-inverso-L de a; e a (·) distribui-se sobre a (+) pela esquerda também:
}
%
\begin{table}[H]
  \begin{dem}
  \end{dem}
  \begin{alvodados}
      \cmnt{}{$(\forall a)[0 + a = a]$}
  \end{alvodados}
\end{table}
%
\begin{table}[H]
  \begin{dem}
      \linha{Seja $a$ um inteiro}
  \end{dem}
  \begin{alvodados}
      \cmnt{$a$}{$0 + a = a$}
  \end{alvodados}
\end{table}
%
\begin{table}[H]
  \begin{dem}
      \linha{Seja $a$ um inteiro}
      \linha{calculamos:}
  \end{dem}
  \begin{alvodados}
      \cmnt{$a$}{$0 + a = a$}
  \end{alvodados}
\end{table}
%
\begin{align*}
  0 + a &= a + 0 \tag*{[(ZA-Com) a 0]}\\
  &= a \tag*{[(ZA-IdR) a]}
\end{align*}
%
\begin{table}[H]
  \begin{dem}
    \linha{seja $a$ inteiro}
    \linha{calculamos:}
  \end{dem}
  \begin{alvodados}
      \cmnt{$a$}{$(-a) + a = 0$}
  \end{alvodados}
\end{table}
%
\begin{align*}
  (-a) + a &= a + (-a) \tag*{[(ZA-Com) a (-a)]}\\
  &= 0 \tag*{[(ZA-InvR)] a}
\end{align*}
%
\begin{table}[H]
  \begin{dem}
    \linha{seja $a$ inteiro}
    \linha{calculamos:}
  \end{dem}
  \begin{alvodados}
      \cmnt{$a$}{$1 \cdot a = a$}
  \end{alvodados}
\end{table}
%
\begin{align*}
  1 \cdot a &= a \cdot 1 \tag*{[(ZM-Com) 1 a]}\\
  &= a \tag*{[(ZM-IdR) a]}
\end{align*}
%
\begin{table}[H]
  \begin{dem}
    \linha{sejam $a, b, d$ inteiros}
    \linha{calculamos:}
  \end{dem}
  \begin{alvodados}
      \cmnt{$a$}{$d \cdot (a+b) = (d\cdot a) + (d \cdot b)$}
      \cmnt{$b$}{}
      \cmnt{$d$}{}
  \end{alvodados}
\end{table}
%
\begin{align*}
  d \cdot (a + b) &= (a + b) \cdot d \tag*{[(ZM-Com) d (a+b)]}\\
  &= (a \cdot d) + (b \cdot d) \tag*{[(Z-distR) a b d]}
\end{align*}
\end{document}