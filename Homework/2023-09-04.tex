\documentclass[a4paper, 12pt]{article}

% PACOTES
\usepackage[utf8]{inputenc} % Caracteres em geral
\usepackage[brazilian]{babel} % Idioma da página
\usepackage[top=2cm, bottom=2cm, left=2cm, right=2cm]{geometry} % margens
\usepackage{fancyhdr} % Cabeçalho
\usepackage{ulem} % Destacar, sublinhar, riscar e etc...
\usepackage{graphicx} % Figuras
\usepackage{booktabs} % Tabelas
\usepackage{multicol} % Multicolunas para tabelas
\usepackage{multirow} % Multilinhas para tabelas
\usepackage{colortbl} % Colorir tabelas
\usepackage[table]{xcolor} % Para usar !25 !50 !75 nas cores
\usepackage{float} % Para manter as tabelas no canto certo
\usepackage{amsmath} % Matrizes
\usepackage{hyperref} % Hyperlinks
\usepackage{tikz} % TikZ - Desenhos e Gráficos
\usepackage{amsmath,amssymb} % Mais símbolos matemáticos
\usepackage[shortlabels]{enumitem} % Usar letras no enumerate

% PREÂMBULOS
\newcounter{linhadem}
\setcounter{linhadem}{0}

\newenvironment{dem}
{
    \rowcolors{2}{gray!25}{white}
    \centering
    \begin{tabular}{p{0.3cm} |p{8cm} |}
    \rowcolor{gray!50}
     & Demonstração\\
}
{
  \end{tabular}
}

\newenvironment{alvodados}
{
    \centering
    \begin{tabular}{p{2.9cm} |p{2.9cm}}
    Dados & Alvo\\
    \hline
}
{
  \end{tabular}
  \setcounter{linhadem}{0}
}

\newcommand{\linha}[1]{%
  \stepcounter{linhadem}
  \thelinhadem & #1 \\
}

\newcommand{\cmnt}[2]{%
  #1 & #2 \\
}

\pagenumbering{Roman}
\tikzstyle{every picture}+=[remember picture,inner xsep=0,inner ysep=0.25ex]

% INÍCIO DO DOCUMENTO
\begin{document}

\section{2023-09-04}

\subsection*{Capítulo 3: até $\theta$3.13}
%
\subsubsection*{EXERCÍCIO x3.1.\\
Para cada axioma que parece favorecendo o lado direito, sua versão esquerda é um
teorema: o 0 é uma (+)-identidade-L; o 1 é uma (·)-identidade-L; para todo a o -a é um
(+)-inverso-L de a; e a (·) distribui-se sobre a (+) pela esquerda também:
}
(ZA-IdL) $(\forall a)[0 + a = a]$
%
\begin{table}[H]
  \begin{dem}
  \end{dem}
  \begin{alvodados}
      \cmnt{}{$(\forall a)[0 + a = a]$}
  \end{alvodados}
\end{table}
%
\begin{table}[H]
  \begin{dem}
      \linha{Seja $a$ um inteiro}
  \end{dem}
  \begin{alvodados}
      \cmnt{$a$}{$0 + a = a$}
  \end{alvodados}
\end{table}
%
\begin{table}[H]
  \begin{dem}
      \linha{Seja $a$ um inteiro}
      \linha{calculamos:}
  \end{dem}
  \begin{alvodados}
      \cmnt{$a$}{$0 + a = a$}
  \end{alvodados}
\end{table}
%
\begin{align*}
  0 + a &= a + 0 \tag*{[(ZA-Com) a 0]}\\
  &= a \tag*{[(ZA-IdR) a]}
\end{align*}
%
(ZA-InvL) $(\forall a)[(-a) + a = 0]$
%
\begin{table}[H]
  \begin{dem}
    \linha{seja $a$ inteiro}
    \linha{calculamos:}
  \end{dem}
  \begin{alvodados}
      \cmnt{$a$}{$(-a) + a = 0$}
  \end{alvodados}
\end{table}
%
\begin{align*}
  (-a) + a &= a + (-a) \tag*{[(ZA-Com) a (-a)]}\\
  &= 0 \tag*{[(ZA-InvR)] a}
\end{align*}
%
(ZM-IdL) $(\forall a)[1 \cdot a = a]$
%
\begin{table}[H]
  \begin{dem}
    \linha{seja $a$ inteiro}
    \linha{calculamos:}
  \end{dem}
  \begin{alvodados}
      \cmnt{$a$}{$1 \cdot a = a$}
  \end{alvodados}
\end{table}
%
\begin{align*}
  1 \cdot a &= a \cdot 1 \tag*{[(ZM-Com) 1 a]}\\
  &= a \tag*{[(ZM-IdR) a]}
\end{align*}
%
(Z-DistL) $(\forall d, a, b)[d \cdot (a + b) = (d \cdot a) + (d \cdot b)]$
%
\begin{table}[H]
  \begin{dem}
    \linha{sejam $a, b, d$ inteiros}
    \linha{calculamos:}
  \end{dem}
  \begin{alvodados}
      \cmnt{$a$}{$d \cdot (a+b) = (d\cdot a) + (d \cdot b)$}
      \cmnt{$b$}{}
      \cmnt{$d$}{}
  \end{alvodados}
\end{table}
%
\begin{align*}
  d \cdot (a + b) &= (a + b) \cdot d \tag*{[(ZM-Com) d (a+b)]}\\
  &= (a \cdot d) + (b \cdot d) \tag*{[(Z-distR) a b d]}
\end{align*}
%
\subsubsection*{EXERCÍCIO x3.2.\\
Demonstre as leis de “passar termo por outro lado”:}
%
$(\forall a,~b,~c )[a + b = c \iff a = c-b]$
%
\begin{table}[H]
  \begin{dem}
    \linha{sejam $a, b, c$ inteiros}
    \linha{split}
  \end{dem}
  \begin{alvodados}
      \cmnt{$a$}{$a + b = c \iff a = c-b$}
      \cmnt{$b$}{}
      \cmnt{$c$}{}
  \end{alvodados}
\end{table}
%
\begin{table}[H]
  \begin{dem}
    \linha{suponha $a + b = c$}
    \linha{calculamos:}
  \end{dem}
  \begin{alvodados}
      \cmnt{$a$}{$a = c-b$}
      \cmnt{$b$}{}
      \cmnt{$c$}{}
      \cmnt{$a + b = c^{[1]}$}{}
  \end{alvodados}
\end{table}
%
\begin{align*}
  c - b &= (a + b) - b \tag*{[1]}\\
  &= a + (b - b) \tag*{[(ZA-Ass) a b -b]}\\
  &= a + 0 \tag*{[(ZA-InvR) b]}\\
  &= a \tag*{[(ZA-IdR) a]}
\end{align*}
%
\begin{table}[H]
  \begin{dem}
    \linha{suponha $a = c - b$}
    \linha{calculamos:}
  \end{dem}
  \begin{alvodados}
      \cmnt{$a$}{$a + b = c$}
      \cmnt{$b$}{}
      \cmnt{$c$}{}
      \cmnt{$a = c - b^{[1]}$}{}
  \end{alvodados}
\end{table}
%
\begin{align*}
  a + b &= (c - b) + b \tag*{[1]}\\
  &= c + ((-b) + b) \tag*{[(ZA-Ass) c -b b]}\\
  &= c + (b + (-b)) \tag*{[(ZA-Com) -b b]}\\
  &= c + 0 \tag*{[(ZA-InvR) b]}\\
  &= c \tag*{[(ZA-IdR) c]}
\end{align*}
%
$(\forall a,~b,~c )[a + b = c \iff b = c-a]$
%
\begin{table}[H]
  \begin{dem}
    \linha{sejam $a, b, c$ inteiros}
    \linha{split}
  \end{dem}
  \begin{alvodados}
      \cmnt{$a$}{$a + b = c \iff b = c-a$}
      \cmnt{$b$}{}
      \cmnt{$c$}{}
  \end{alvodados}
\end{table}
%
\begin{table}[H]
  \begin{dem}
    \linha{suponha $a + b = c$}
    \linha{calculamos:}
  \end{dem}
  \begin{alvodados}
      \cmnt{$a$}{$b = c-a$}
      \cmnt{$b$}{}
      \cmnt{$c$}{}
      \cmnt{$a + b = c^{[1]}$}{}
  \end{alvodados}
\end{table}
%
\begin{align*}
  c - a &= (a + b) - a \tag*{[1]}\\
  &= (b + a) - a \tag*{[(ZA-Com) a b]}\\
  &= b + (a - a) \tag*{[(ZA-Ass) b a -a]}\\
  &= b + 0 \tag*{[(ZA-InvR)]}\\
  &= b \tag*{[(ZA-IdR) b]}
\end{align*}
%
\begin{table}[H]
  \begin{dem}
    \linha{suponha $b = c - a$}
    \linha{calculamos:}
  \end{dem}
  \begin{alvodados}
      \cmnt{$a$}{$a + b = c$}
      \cmnt{$b$}{}
      \cmnt{$c$}{}
      \cmnt{$b = c - a^{[1]}$}{}
  \end{alvodados}
\end{table}
%
\begin{align*}
  a + b &= a + (c - a) \tag*{[1]}\\
  &= a + ((-a) + c) \tag*{[(ZA-Com) c -a]}\\
  &= a + (-a) + c \tag*{{[(ZA-Ass) a -a c]}}\\
  &= 0 + c \tag*{[(ZA-InvR) a]}\\
  &= c \tag*{[(ZA-IdL) c]}
\end{align*}
%
$(\forall a,~b )[a = b \iff a - b = 0]$
%
\begin{table}[H]
  \begin{dem}
    \linha{sejam $a,~b$ inteiros}
    \linha{split}
  \end{dem}
  \begin{alvodados}
      \cmnt{$a$}{$a = b \iff a - b = 0$}
      \cmnt{$b$}{}
  \end{alvodados}
\end{table}
%
\begin{table}[H]
  \begin{dem}
    \linha{Suponha $a=b$}
    \linha{calculamos:}
  \end{dem}
  \begin{alvodados}
    \cmnt{$a$}{$a - b = 0$}
    \cmnt{$b$}{}
    \cmnt{$a=b^{[1]}$}{}
  \end{alvodados}
\end{table}
%
\begin{align*}
  a - b &= a - a \tag*{[1]}\\
  &= 0 \tag*{[(ZA-InvR) a]}
\end{align*}
%
\begin{table}[H]
  \begin{dem}
    \linha{Suponha $a-b=0$}
    \linha{calculamos:}
  \end{dem}
  \begin{alvodados}
    \cmnt{$a$}{$a = b$}
    \cmnt{$b$}{}
    \cmnt{$a-b=0^{[1]}$}{}
  \end{alvodados}
\end{table}
%
\begin{align*}
  b &= b + 0 \tag*{[(ZA-IdR) a]}\\
  &= b + (a - b) \tag*{[1]}\\
  &= b + ((-b) + a) \tag*{[(ZA-Com) a -b]}\\
  &= (b + (-b)) + a \tag*{[(ZA-Ass) b -b a]}\\
  &= 0 + a \tag*{[(ZA-InvR) b]}\\
  &= a \tag*{[(ZA-IdL) a]}  
\end{align*}

\end{document}