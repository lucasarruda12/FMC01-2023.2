\documentclass[a4paper, 12pt]{article}

% PACOTES
\usepackage[utf8]{inputenc} % Caracteres em geral
\usepackage[brazilian]{babel} % Idioma da página
\usepackage[top=2cm, bottom=2cm, left=2cm, right=2cm]{geometry} % margens
\usepackage{fancyhdr} % Cabeçalho
\usepackage{ulem} % Destacar, sublinhar, riscar e etc...
\usepackage{graphicx} % Figuras
\usepackage{booktabs} % Tabelas
\usepackage{multicol} % Multicolunas para tabelas
\usepackage{multirow} % Multilinhas para tabelas
\usepackage{colortbl} % Colorir tabelas
\usepackage[table]{xcolor} % Para usar !25 !50 !75 nas cores
\usepackage{float} % Para manter as tabelas no canto certo
\usepackage{amsmath} % Matrizes
\usepackage{hyperref} % Hyperlinks
\usepackage{tikz} % TikZ - Desenhos e Gráficos
\usepackage{amsmath,amssymb} % Mais símbolos matemáticos
\usepackage[shortlabels]{enumitem} % Usar letras no enumerate

% PREÂMBULOS
\newcounter{linhadem}
\setcounter{linhadem}{0}

\newenvironment{dem}
{
    \rowcolors{2}{gray!25}{white}
    \centering
    \begin{tabular}{p{0.3cm} |p{8cm} |}
    \rowcolor{gray!50}
     & Demonstração\\
}
{
  \end{tabular}
}

\newenvironment{alvodados}
{
    \centering
    \begin{tabular}{p{2.9cm} |p{2.9cm}}
    Dados & Alvo\\
    \hline
}
{
  \end{tabular}
  \setcounter{linhadem}{0}
}

\newcommand{\linha}[1]{%
  \stepcounter{linhadem}
  \thelinhadem & #1 \\
}

\newcommand{\cmnt}[2]{%
  #1 & #2 \\
}

\pagenumbering{Roman}
\tikzstyle{every picture}+=[remember picture,inner xsep=0,inner ysep=0.25ex]

% INÍCIO DO DOCUMENTO
\begin{document}

\section{2023-09-02}

\subsection*{Jogue o NNG. Tente completar o mais rápido possivel; sua entregá pode acabar sendo a Prova 2.2. Veja o FAQ relevante também.}

\subsection*{Defina (sem consulta) a adição, a multiplicação, e a exponenciação.}
%
Adição ($\forall n:Nat$, $\forall m:Nat$ ):
\begin{align*}
    \text{Nat} \times \text{Nat} &\to \text{Nat}\\
    n + O &= n \tag*{[a.1]}\\
    n + Sm &= S(n + m) \tag*{[a.2]}
\end{align*}
%
Multiplicação ($\forall n:Nat$, $\forall m:Nat$ ):
\begin{align*}
    \text{Nat} \times \text{Nat} &\to \text{Nat}\\
    n \cdot O &= O \tag*{[m.1]}\\
    n \cdot Sm &= n + n \cdot m \tag*{[m.2]}
\end{align*}
%
Exponenciação ($\forall n:Nat$, $\forall m:Nat$ ):
\begin{align*}
    \text{Nat} \times \text{Nat} &\to \text{Nat}\\
    n^O &= SO \tag*{[e.1]}\\
    n^{Sm} &= n \cdot n^m \tag*{[e.2]}
\end{align*}

\subsection*{Defina uma função double que retorna o dobro da sua entrada.}
%
Double ($\forall n:Nat$):
\begin{align*}
    \text{Nat} &\to \text{Nat}\\
    double(n) &= n + n \tag*{[d.1]}
\end{align*}
%
\subsection*{Calcule os:}
%
\subsubsection*{$double(double(S O))$}
%
\begin{align*}
    double(double(S O)) &= double(SO + SO) \tag*{[d.1; SO]}\\
    &= double(S(SO + O)) \tag*{[a.2; SO, O]}\\
    &= double(SSO) \tag*{[a.1; SO]}\\
    &= SSO + SSO \tag*{[d.1; SSO]}\\
    &= S(SSO + SO) \tag*{[a.2; SSO, SO]}\\
    &= S(S(SSO + O)) \tag*{[a.2; SSO, O]}\\
    &= SSSSO \tag*{[a.1; SSO]}\\
\end{align*}
%
\subsubsection*{$double ((S O) + (S (S O)))$}
%
\begin{align*}
    double ((S O) + (S (S O))) &= double(S(SO + SO)) \tag*{[a.2; SO, SO]}\\
    &= double(S(S(SO + O))) \tag*{[a.2; SO, O]}\\
    &= double(SSSO) \tag*{[a.1; SO]}\\
    &= SSSO + SSSO \tag*{[d.1; SSSO]}\\
    &= S(SSSO + SSO) \tag*{[a.2; SSSO, SSO]}\\
    &= S(S(SSSO + SO)) \tag*{[a.2; SSSO, SO]}\\
    &= S(S(S(SSSO + O))) \tag*{[a.2; SSSO, O]}\\
    &= SSSSSSO \tag*{[a.1; SSSO]}
\end{align*}
%
\subsection*{Demonstre uma das: (·)-idL, (·)-idR}
%
\begin{table}[H]
    \begin{dem}
    \end{dem}
    \begin{alvodados}
        \cmnt{}{$(\exists i)(\forall a)[a \cdot i = a]$}
    \end{alvodados}
\end{table}
%
\begin{table}[H]
    \begin{dem}
        \linha{escolha $SO$}
    \end{dem}
    \begin{alvodados}
      \cmnt{$SO$}{$(\forall a)[a \cdot SO = a]$}
    \end{alvodados}
\end{table}
%
\begin{table}[H]
    \begin{dem}
        \linha{escolha $SO$}
        \linha{seja $a$ Nat}
    \end{dem}
    \begin{alvodados}
      \cmnt{$SO$}{$a \cdot SO = a$}
      \cmnt{$a: Nat$}{}
    \end{alvodados}
\end{table}
%
\begin{table}[H]
    \begin{dem}
        \linha{escolha $SO$}
        \linha{seja $a$ Nat}
        \linha{calculamos:}
    \end{dem}
    \begin{alvodados}
      \cmnt{$SO$}{$a \cdot SO = a$}
      \cmnt{$a: Nat$}{}
    \end{alvodados}
\end{table}
%
\begin{align*}
    a \cdot SO &= a + a \cdot O \tag*{[[m.2] a O]}\\
    &= a + O \tag*{[[m.1] a]}\\
    &= a \tag*{[[a,.1] a]}
\end{align*}
%
\subsection*{Demonstre uma das: (\^{})-idL, (\^{})-idR}
%
\begin{table}[H]
    \begin{dem}
    \end{dem}
    \begin{alvodados}
        \cmnt{}{$(\exists i)(\forall a)[a ^i = a]$}
    \end{alvodados}
\end{table}
%
\begin{table}[H]
    \begin{dem}
        \linha{Escolha $SO$}
    \end{dem}
    \begin{alvodados}
        \cmnt{$SO$}{$(\forall a)[a ^i = a]$}
    \end{alvodados}
\end{table}
%
\begin{table}[H]
    \begin{dem}
        \linha{Escolha $SO$}
        \linha{Seja $a:Nat$}
    \end{dem}
    \begin{alvodados}
        \cmnt{$SO$}{$a ^i = a$}
        \cmnt{$a:Nat$}{}
    \end{alvodados}
\end{table}
%
\begin{table}[H]
    \begin{dem}
        \linha{Escolha $SO$}
        \linha{Seja $a:Nat$}
        \linha{Calculamos:}
    \end{dem}
    \begin{alvodados}
        \cmnt{$SO$}{$a ^i = a$}
        \cmnt{$a:Nat$}{}
    \end{alvodados}
\end{table}
%
\begin{align*}
    a ^{SO} &= a \cdot a ^ O \tag*{[[e.2] a O]}\\
    &= a \cdot SO \tag*{[[e.1] a]}\\
    &= a + a \cdot O \tag*{[[m.2] a O]}\\
    &= a + O \tag*{[[m.1] a]}\\
    &= a \tag*{[[a.1] a]}
\end{align*}
%
\subsection*{Defina a função pred : Nat → Nat de “predecessor”, onde consideramos que o predecessor de zero é o próprio zero mesmo: AVISO: a maioria das vezes que um aluno de 2022.1 e de 2022.2 usou a pred em outras definições, estava errando}
%
Pred ($\forall n:Nat$):
\begin{align*}
    Nat &\to Nat\\
    pred(O) &= O \tag*{[pd.1]}\\
    pred(Sn) &= n \tag*{[pd.2]}
\end{align*}
%
\subsection*{Defina as funções:}
%
($\forall n:int$, $\forall m:int$):
%
\subsubsection*{fact : Nat $\to$ Nat}
%
\begin{align*}
    Nat &\to Nat\\
    O! &= SO\\
    Sn! &= Sn \cdot n!
\end{align*}
%
\subsubsection*{fib  : Nat $\to$ Nat}
%
\begin{align*}
    Nat &\to Nat\\
    fib(O) &= O\\
    fib(SO) &= SO\\
    fib(SSn) &= fib(Sn) + fib(n)
\end{align*}
%
\subsubsection*{min  : Nat $\times$ Nat $\to$ Nat}
%
\begin{align*}
    Nat \times Nat &\to Nat\\
    min(n, O) &= O\\
    min(O, n) &= O\\
    min(Sn, Sm) &= S(min(n, m))
\end{align*}
%
\subsubsection*{min  : Nat $\times$ Nat $\to$ Nat}
%
\begin{align*}
    Nat \times Nat &\to Nat\\
    max(n, O) &= n\\
    max(O, n) &= n\\
    max(Sn, Sm) &= S(max(n, m))
\end{align*}
%

\end{document}